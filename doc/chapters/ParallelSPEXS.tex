\chapter{Parallelization}

Here we describe full parallelization of the previous 
algorithm. The fully parallelized code can be seen in the appendix.

\section{Process}

The main process of the algorithm as described by dataflow:

\begin{verbatim}
									     q1   => |filter|
	IN POOL == take q ==> |EXTENDER| <   q2   => |filter| > => OUT POOL
		|								 q3   => |filter|
		|							      v
		|< =======================     |filter|
\end{verbatim}

The easiest thing here to parallelize is the extender since it's interaction
can be seen as a separate unit.

\section{Horizontal scaling}

If instead of in/out pool we had several the algorithm can still work, if we
have single take / put process to decide which actual pool to use.

We can use several extenders that work in parallel without problems since they do not need information about other querys nor input/output pools.

DIAGRAM

\section{Filtering}

This is the only place where we may need the whole information about the query.

Although most operations can be parallelized with map reduce.

\section{Extending}

We still can also do the extending process in parallel:

\begin{verbatim}
func extender(query) {
	// find all next positions
	nexts = (map next query.matches)
	matches = (group nexts by token)
	result = (map newquery matches)
	other = (map #(union (matches %1)) groups)
	return result union other
}
\end{verbatim}

\section{Dataset partitioning}

If we look at where we need synchronization points
if we partition the dataset. Whole parts of 
extension still apply for parts of dataset.

Synchronization is only needed for filtering.
We don't need to know the whole query but just
information about parts to make a decision whether
to filter or not.

